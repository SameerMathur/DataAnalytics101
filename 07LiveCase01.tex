% Options for packages loaded elsewhere
\PassOptionsToPackage{unicode}{hyperref}
\PassOptionsToPackage{hyphens}{url}
\PassOptionsToPackage{dvipsnames,svgnames,x11names}{xcolor}
%
\documentclass[
  letterpaper,
  DIV=11,
  numbers=noendperiod]{scrreport}

\usepackage{amsmath,amssymb}
\usepackage{iftex}
\ifPDFTeX
  \usepackage[T1]{fontenc}
  \usepackage[utf8]{inputenc}
  \usepackage{textcomp} % provide euro and other symbols
\else % if luatex or xetex
  \usepackage{unicode-math}
  \defaultfontfeatures{Scale=MatchLowercase}
  \defaultfontfeatures[\rmfamily]{Ligatures=TeX,Scale=1}
\fi
\usepackage{lmodern}
\ifPDFTeX\else  
    % xetex/luatex font selection
\fi
% Use upquote if available, for straight quotes in verbatim environments
\IfFileExists{upquote.sty}{\usepackage{upquote}}{}
\IfFileExists{microtype.sty}{% use microtype if available
  \usepackage[]{microtype}
  \UseMicrotypeSet[protrusion]{basicmath} % disable protrusion for tt fonts
}{}
\makeatletter
\@ifundefined{KOMAClassName}{% if non-KOMA class
  \IfFileExists{parskip.sty}{%
    \usepackage{parskip}
  }{% else
    \setlength{\parindent}{0pt}
    \setlength{\parskip}{6pt plus 2pt minus 1pt}}
}{% if KOMA class
  \KOMAoptions{parskip=half}}
\makeatother
\usepackage{xcolor}
\setlength{\emergencystretch}{3em} % prevent overfull lines
\setcounter{secnumdepth}{-\maxdimen} % remove section numbering
% Make \paragraph and \subparagraph free-standing
\ifx\paragraph\undefined\else
  \let\oldparagraph\paragraph
  \renewcommand{\paragraph}[1]{\oldparagraph{#1}\mbox{}}
\fi
\ifx\subparagraph\undefined\else
  \let\oldsubparagraph\subparagraph
  \renewcommand{\subparagraph}[1]{\oldsubparagraph{#1}\mbox{}}
\fi

\usepackage{color}
\usepackage{fancyvrb}
\newcommand{\VerbBar}{|}
\newcommand{\VERB}{\Verb[commandchars=\\\{\}]}
\DefineVerbatimEnvironment{Highlighting}{Verbatim}{commandchars=\\\{\}}
% Add ',fontsize=\small' for more characters per line
\usepackage{framed}
\definecolor{shadecolor}{RGB}{241,243,245}
\newenvironment{Shaded}{\begin{snugshade}}{\end{snugshade}}
\newcommand{\AlertTok}[1]{\textcolor[rgb]{0.68,0.00,0.00}{#1}}
\newcommand{\AnnotationTok}[1]{\textcolor[rgb]{0.37,0.37,0.37}{#1}}
\newcommand{\AttributeTok}[1]{\textcolor[rgb]{0.40,0.45,0.13}{#1}}
\newcommand{\BaseNTok}[1]{\textcolor[rgb]{0.68,0.00,0.00}{#1}}
\newcommand{\BuiltInTok}[1]{\textcolor[rgb]{0.00,0.23,0.31}{#1}}
\newcommand{\CharTok}[1]{\textcolor[rgb]{0.13,0.47,0.30}{#1}}
\newcommand{\CommentTok}[1]{\textcolor[rgb]{0.37,0.37,0.37}{#1}}
\newcommand{\CommentVarTok}[1]{\textcolor[rgb]{0.37,0.37,0.37}{\textit{#1}}}
\newcommand{\ConstantTok}[1]{\textcolor[rgb]{0.56,0.35,0.01}{#1}}
\newcommand{\ControlFlowTok}[1]{\textcolor[rgb]{0.00,0.23,0.31}{#1}}
\newcommand{\DataTypeTok}[1]{\textcolor[rgb]{0.68,0.00,0.00}{#1}}
\newcommand{\DecValTok}[1]{\textcolor[rgb]{0.68,0.00,0.00}{#1}}
\newcommand{\DocumentationTok}[1]{\textcolor[rgb]{0.37,0.37,0.37}{\textit{#1}}}
\newcommand{\ErrorTok}[1]{\textcolor[rgb]{0.68,0.00,0.00}{#1}}
\newcommand{\ExtensionTok}[1]{\textcolor[rgb]{0.00,0.23,0.31}{#1}}
\newcommand{\FloatTok}[1]{\textcolor[rgb]{0.68,0.00,0.00}{#1}}
\newcommand{\FunctionTok}[1]{\textcolor[rgb]{0.28,0.35,0.67}{#1}}
\newcommand{\ImportTok}[1]{\textcolor[rgb]{0.00,0.46,0.62}{#1}}
\newcommand{\InformationTok}[1]{\textcolor[rgb]{0.37,0.37,0.37}{#1}}
\newcommand{\KeywordTok}[1]{\textcolor[rgb]{0.00,0.23,0.31}{#1}}
\newcommand{\NormalTok}[1]{\textcolor[rgb]{0.00,0.23,0.31}{#1}}
\newcommand{\OperatorTok}[1]{\textcolor[rgb]{0.37,0.37,0.37}{#1}}
\newcommand{\OtherTok}[1]{\textcolor[rgb]{0.00,0.23,0.31}{#1}}
\newcommand{\PreprocessorTok}[1]{\textcolor[rgb]{0.68,0.00,0.00}{#1}}
\newcommand{\RegionMarkerTok}[1]{\textcolor[rgb]{0.00,0.23,0.31}{#1}}
\newcommand{\SpecialCharTok}[1]{\textcolor[rgb]{0.37,0.37,0.37}{#1}}
\newcommand{\SpecialStringTok}[1]{\textcolor[rgb]{0.13,0.47,0.30}{#1}}
\newcommand{\StringTok}[1]{\textcolor[rgb]{0.13,0.47,0.30}{#1}}
\newcommand{\VariableTok}[1]{\textcolor[rgb]{0.07,0.07,0.07}{#1}}
\newcommand{\VerbatimStringTok}[1]{\textcolor[rgb]{0.13,0.47,0.30}{#1}}
\newcommand{\WarningTok}[1]{\textcolor[rgb]{0.37,0.37,0.37}{\textit{#1}}}

\providecommand{\tightlist}{%
  \setlength{\itemsep}{0pt}\setlength{\parskip}{0pt}}\usepackage{longtable,booktabs,array}
\usepackage{calc} % for calculating minipage widths
% Correct order of tables after \paragraph or \subparagraph
\usepackage{etoolbox}
\makeatletter
\patchcmd\longtable{\par}{\if@noskipsec\mbox{}\fi\par}{}{}
\makeatother
% Allow footnotes in longtable head/foot
\IfFileExists{footnotehyper.sty}{\usepackage{footnotehyper}}{\usepackage{footnote}}
\makesavenoteenv{longtable}
\usepackage{graphicx}
\makeatletter
\def\maxwidth{\ifdim\Gin@nat@width>\linewidth\linewidth\else\Gin@nat@width\fi}
\def\maxheight{\ifdim\Gin@nat@height>\textheight\textheight\else\Gin@nat@height\fi}
\makeatother
% Scale images if necessary, so that they will not overflow the page
% margins by default, and it is still possible to overwrite the defaults
% using explicit options in \includegraphics[width, height, ...]{}
\setkeys{Gin}{width=\maxwidth,height=\maxheight,keepaspectratio}
% Set default figure placement to htbp
\makeatletter
\def\fps@figure{htbp}
\makeatother

\usepackage{booktabs}
\usepackage{longtable}
\usepackage{array}
\usepackage{multirow}
\usepackage{wrapfig}
\usepackage{float}
\usepackage{colortbl}
\usepackage{pdflscape}
\usepackage{tabu}
\usepackage{threeparttable}
\usepackage{threeparttablex}
\usepackage[normalem]{ulem}
\usepackage{makecell}
\usepackage{xcolor}
\KOMAoption{captions}{tableheading}
\titlehead{
  \begin{center}
    \includegraphics[width=5in]{FINALIZED BOOK COVER.png}
  \end{center}
}
\makeatletter
\makeatother
\makeatletter
\makeatother
\makeatletter
\@ifpackageloaded{caption}{}{\usepackage{caption}}
\AtBeginDocument{%
\ifdefined\contentsname
  \renewcommand*\contentsname{Table of contents}
\else
  \newcommand\contentsname{Table of contents}
\fi
\ifdefined\listfigurename
  \renewcommand*\listfigurename{List of Figures}
\else
  \newcommand\listfigurename{List of Figures}
\fi
\ifdefined\listtablename
  \renewcommand*\listtablename{List of Tables}
\else
  \newcommand\listtablename{List of Tables}
\fi
\ifdefined\figurename
  \renewcommand*\figurename{Figure}
\else
  \newcommand\figurename{Figure}
\fi
\ifdefined\tablename
  \renewcommand*\tablename{Table}
\else
  \newcommand\tablename{Table}
\fi
}
\@ifpackageloaded{float}{}{\usepackage{float}}
\floatstyle{ruled}
\@ifundefined{c@chapter}{\newfloat{codelisting}{h}{lop}}{\newfloat{codelisting}{h}{lop}[chapter]}
\floatname{codelisting}{Listing}
\newcommand*\listoflistings{\listof{codelisting}{List of Listings}}
\makeatother
\makeatletter
\@ifpackageloaded{caption}{}{\usepackage{caption}}
\@ifpackageloaded{subcaption}{}{\usepackage{subcaption}}
\makeatother
\makeatletter
\@ifpackageloaded{tcolorbox}{}{\usepackage[skins,breakable]{tcolorbox}}
\makeatother
\makeatletter
\@ifundefined{shadecolor}{\definecolor{shadecolor}{rgb}{.97, .97, .97}}
\makeatother
\makeatletter
\makeatother
\makeatletter
\makeatother
\ifLuaTeX
  \usepackage{selnolig}  % disable illegal ligatures
\fi
\IfFileExists{bookmark.sty}{\usepackage{bookmark}}{\usepackage{hyperref}}
\IfFileExists{xurl.sty}{\usepackage{xurl}}{} % add URL line breaks if available
\urlstyle{same} % disable monospaced font for URLs
\hypersetup{
  colorlinks=true,
  linkcolor={blue},
  filecolor={Maroon},
  citecolor={Blue},
  urlcolor={Blue},
  pdfcreator={LaTeX via pandoc}}

\author{}
\date{}

\begin{document}
\ifdefined\Shaded\renewenvironment{Shaded}{\begin{tcolorbox}[breakable, frame hidden, boxrule=0pt, sharp corners, interior hidden, borderline west={3pt}{0pt}{shadecolor}, enhanced]}{\end{tcolorbox}}\fi

\hypertarget{live-case-sp500-1-of-3}{%
\chapter{Live Case: S\&P500 (1 of 3)}\label{live-case-sp500-1-of-3}}

\emph{Aug 18, 2023.}

\hypertarget{overview-of-the-sp-500}{%
\section{Overview of the S\&P 500}\label{overview-of-the-sp-500}}

\begin{enumerate}
\def\labelenumi{\arabic{enumi}.}
\item
  The S\&P 500, also called the Standard \& Poor's 500, is a stock
  market index that tracks the performance of 500 major publicly traded
  companies listed on U.S. stock exchanges. It serves as a widely
  accepted benchmark for assessing the overall health and performance of
  the U.S. stock market.
\item
  S\&P Dow Jones Indices, a division of S\&P Global, is responsible for
  maintaining the index. The selection of companies included in the S\&P
  500 is determined by a committee, considering factors such as market
  capitalization, liquidity, and industry representation.
\item
  The S\&P is a float-weighted index, meaning the market capitalization
  of the companies in the index are adjusted by the number of shares
  available for public trading. {[}1{]}
\item
  The performance of the S\&P 500 is frequently used to gauge the
  broader stock market and is commonly referenced by investors,
  analysts, and financial media. It provides a snapshot of how large-cap
  U.S. stocks are faring and is considered a reliable indicator of
  overall market sentiment.
\item
  Companies that relatively have the highest Market Capitalization
  include Apple, Microsoft, Amazon, Google's parent company, Alphabet.
  {[}1{]}
\end{enumerate}

\hypertarget{sp-500-data---preliminary-setup}{%
\section{S\&P 500 Data - Preliminary
Setup}\label{sp-500-data---preliminary-setup}}

\begin{enumerate}
\def\labelenumi{\arabic{enumi}.}
\item
  We will analyze a real-world, recent dataset containing information
  about the S\&P500 stocks. The dataset is located in a Google Sheet
  {[}2{]}
\item
  Load necessary libraries, suppressing annoying start up messages.
\end{enumerate}

\begin{Shaded}
\begin{Highlighting}[]
\FunctionTok{library}\NormalTok{(knitr)}
\FunctionTok{suppressPackageStartupMessages}\NormalTok{(}\FunctionTok{library}\NormalTok{(dplyr))}
\FunctionTok{suppressPackageStartupMessages}\NormalTok{(}\FunctionTok{library}\NormalTok{(kableExtra))}
\FunctionTok{suppressPackageStartupMessages}\NormalTok{(}\FunctionTok{library}\NormalTok{(lubridate))}
\FunctionTok{library}\NormalTok{(gsheet)}
\end{Highlighting}
\end{Shaded}

\hypertarget{read-the-sp500-data-from-a-google-sheet-into-a-tibble}{%
\subsection{Read the S\&P500 data from a Google Sheet into a
tibble}\label{read-the-sp500-data-from-a-google-sheet-into-a-tibble}}

\begin{enumerate}
\def\labelenumi{\arabic{enumi}.}
\item
  The complete URL is\\
  https://docs.google.com/spreadsheets/d/11ahk9uWxBkDqrhNm7qYmiTwrlSC53N1zvXYfv7ttOCM/
\item
  The Google Sheet ID is:
  \texttt{11ahk9uWxBkDqrhNm7qYmiTwrlSC53N1zvXYfv7ttOCM}. We can use the
  function \texttt{gsheet2tbl} in package \texttt{gsheet} to read the
  Google Sheet into a tibble or dataframe, as demonstrated in the
  following code.
\end{enumerate}

\begin{Shaded}
\begin{Highlighting}[]
\CommentTok{\# Read S\&P500 stock data present in a Google Sheet.}
\FunctionTok{library}\NormalTok{(gsheet)}
\NormalTok{prefix }\OtherTok{\textless{}{-}} \StringTok{"https://docs.google.com/spreadsheets/d/"}
\NormalTok{sheetID }\OtherTok{\textless{}{-}} \StringTok{"11ahk9uWxBkDqrhNm7qYmiTwrlSC53N1zvXYfv7ttOCM"}
\NormalTok{url500 }\OtherTok{\textless{}{-}} \FunctionTok{paste}\NormalTok{(prefix,sheetID) }\CommentTok{\# Form the URL to connect to}
\NormalTok{sp500 }\OtherTok{\textless{}{-}} \FunctionTok{gsheet2tbl}\NormalTok{(url500) }\CommentTok{\# Read it into a tibble called sp500}
\end{Highlighting}
\end{Shaded}

\begin{enumerate}
\def\labelenumi{\arabic{enumi}.}
\setcounter{enumi}{2}
\tightlist
\item
  \textbf{Discussion}:
\end{enumerate}

\begin{itemize}
\item
  We load the \texttt{gsheet} library to fetch Google Sheets as data
  frames.
\item
  We create the URL for the specific Google Sheet containing S\&P500
  stock data.
\item
  Using the \texttt{gsheet2tbl} function, we read the data from this
  Google Sheet into a tibble named \texttt{sp500}.
\end{itemize}

\hypertarget{date}{%
\subsection{Date}\label{date}}

\begin{enumerate}
\def\labelenumi{\arabic{enumi}.}
\setcounter{enumi}{3}
\tightlist
\item
  We identify the date of the data.
\end{enumerate}

\begin{Shaded}
\begin{Highlighting}[]
\NormalTok{d1 }\OtherTok{\textless{}{-}} \FunctionTok{unique}\NormalTok{(sp500}\SpecialCharTok{$}\NormalTok{Date)}
\NormalTok{d2 }\OtherTok{\textless{}{-}} \FunctionTok{mdy}\NormalTok{(d1)}
\NormalTok{date }\OtherTok{\textless{}{-}} \FunctionTok{format}\NormalTok{(d2, }\StringTok{"\%d \%B \%Y"}\NormalTok{) }\CommentTok{\#Save the date}
\end{Highlighting}
\end{Shaded}

\begin{itemize}
\tightlist
\item
  We have 503 stocks of the S\&P500 in our dataset, as of 13 September
  2023.
\end{itemize}

\hypertarget{reviewing-the-data-types}{%
\section{Reviewing the data types}\label{reviewing-the-data-types}}

\begin{enumerate}
\def\labelenumi{\arabic{enumi}.}
\tightlist
\item
  We need to understand the different data columns and their data types.
\end{enumerate}

\begin{Shaded}
\begin{Highlighting}[]
\NormalTok{dataTypes }\OtherTok{\textless{}{-}} \FunctionTok{tibble}\NormalTok{(}
  \AttributeTok{ColumnName =} \FunctionTok{names}\NormalTok{(sp500),}
  \AttributeTok{DataType =} \FunctionTok{sapply}\NormalTok{(sp500, class)}
\NormalTok{)}
\FunctionTok{kable}\NormalTok{(dataTypes, }\AttributeTok{caption =} \StringTok{"Data Columns and their data types"}\NormalTok{) }
\end{Highlighting}
\end{Shaded}

\begin{table}

\caption{Data Columns and their data types}
\centering
\begin{tabular}[t]{l|l}
\hline
ColumnName & DataType\\
\hline
Date & character\\
\hline
Stock & character\\
\hline
Description & character\\
\hline
Sector & character\\
\hline
Industry & character\\
\hline
Market Capitalization & numeric\\
\hline
Price & numeric\\
\hline
52 Week Low & numeric\\
\hline
52 Week High & numeric\\
\hline
Return on Equity (TTM) & numeric\\
\hline
Return on Assets (TTM) & numeric\\
\hline
Return on Invested Capital (TTM) & numeric\\
\hline
Gross Margin (TTM) & numeric\\
\hline
Operating Margin (TTM) & numeric\\
\hline
Net Margin (TTM) & numeric\\
\hline
Price to Earnings Ratio (TTM) & numeric\\
\hline
Price to Book (FY) & numeric\\
\hline
Enterprise Value/EBITDA (TTM) & numeric\\
\hline
EBITDA (TTM) & numeric\\
\hline
EPS Diluted (TTM) & numeric\\
\hline
EBITDA (TTM YoY Growth) & numeric\\
\hline
EBITDA (Quarterly YoY Growth) & numeric\\
\hline
EPS Diluted (TTM YoY Growth) & numeric\\
\hline
EPS Diluted (Quarterly YoY Growth) & numeric\\
\hline
Price to Free Cash Flow (TTM) & numeric\\
\hline
Free Cash Flow (TTM YoY Growth) & numeric\\
\hline
Free Cash Flow (Quarterly YoY Growth) & numeric\\
\hline
Debt to Equity Ratio (MRQ) & numeric\\
\hline
Current Ratio (MRQ) & numeric\\
\hline
Quick Ratio (MRQ) & numeric\\
\hline
Dividend Yield Forward & numeric\\
\hline
Dividends per share (Annual YoY Growth) & numeric\\
\hline
Price to Sales (FY) & numeric\\
\hline
Revenue (TTM YoY Growth) & numeric\\
\hline
Revenue (Quarterly YoY Growth) & numeric\\
\hline
Technical Rating & character\\
\hline
\end{tabular}
\end{table}

\begin{enumerate}
\def\labelenumi{\arabic{enumi}.}
\setcounter{enumi}{2}
\tightlist
\item
  \textbf{Discussion}:
\end{enumerate}

\begin{itemize}
\item
  We create a tibble named \texttt{dataTypes} that lists the column
  names from the \texttt{sp500} tibble and the corresponding data types
  for each column.
\item
  We use \texttt{kable} to display this information as a neatly
  formatted table.
\item
  Note that we use the \texttt{sapply()} function to apply the
  \texttt{class()} function to every column in the \texttt{sp500}
  dataset. This will retrieve the data type of each column.
\end{itemize}

\begin{enumerate}
\def\labelenumi{\arabic{enumi}.}
\setcounter{enumi}{3}
\tightlist
\item
  \textbf{Data Columns}:
\end{enumerate}

\begin{itemize}
\item
  The columns labeled \texttt{Date}, \texttt{Stock},
  \texttt{Description}, \texttt{Sector}, and \texttt{Industry} are
  character columns. They respectively represent the date, stock ticker
  symbol, description, sector, and industry of each S\&P500 stock.
\item
  Columns such as \texttt{Market\ Capitalization}, \texttt{Price},
  \texttt{52\ Week\ Low}, \texttt{52\ Week\ High}, and other numeric
  columns contain diverse financial metrics and stock prices related to
  the S\&P500 stocks.
\item
  The column labeled \texttt{Technical\ Rating} gives a Buy / Sell
  technical rating for each stock.
\end{itemize}

\hypertarget{remove-rows-containing-no-data-or-null-values}{%
\subsection{Remove Rows containing no data or Null
values}\label{remove-rows-containing-no-data-or-null-values}}

\begin{itemize}
\tightlist
\item
  We check if the ``Stock'' column in the sp500 dataframe contains any
  null or blank values and removes them.
\end{itemize}

\begin{Shaded}
\begin{Highlighting}[]
\NormalTok{hasNull }\OtherTok{\textless{}{-}} \FunctionTok{any}\NormalTok{(sp500}\SpecialCharTok{$}\NormalTok{Stock }\SpecialCharTok{==} \StringTok{""} \SpecialCharTok{|} \FunctionTok{is.null}\NormalTok{(sp500}\SpecialCharTok{$}\NormalTok{Stock)) }\CommentTok{\# Check for blank or null values }
\ControlFlowTok{if}\NormalTok{ (hasNull) \{ }
\NormalTok{    sp500 }\OtherTok{\textless{}{-}}\NormalTok{ sp500[}\SpecialCharTok{!}\NormalTok{(}\FunctionTok{is.null}\NormalTok{(sp500}\SpecialCharTok{$}\NormalTok{Stock) }\SpecialCharTok{|}\NormalTok{ sp500}\SpecialCharTok{$}\NormalTok{Stock }\SpecialCharTok{==} \StringTok{""}\NormalTok{), ] }\CommentTok{\# Remove any null or blank values}
\NormalTok{\}}
\end{Highlighting}
\end{Shaded}

\hypertarget{sp500-sectors}{%
\subsection{S\&P500 Sectors}\label{sp500-sectors}}

\begin{enumerate}
\def\labelenumi{\arabic{enumi}.}
\tightlist
\item
  The S\&P500 shares are divided into multiple sectors. Each stock
  belongs to a unique sector. Thus, it makes sense to model
  \texttt{Sector} as a \texttt{factor}.
\end{enumerate}

\begin{Shaded}
\begin{Highlighting}[]
\NormalTok{sp500}\SpecialCharTok{$}\NormalTok{Sector }\OtherTok{\textless{}{-}} \FunctionTok{as.factor}\NormalTok{(sp500}\SpecialCharTok{$}\NormalTok{Sector)}
\end{Highlighting}
\end{Shaded}

\begin{enumerate}
\def\labelenumi{\arabic{enumi}.}
\setcounter{enumi}{1}
\tightlist
\item
  The \texttt{table()} function allows us to count how many stocks are
  part of each sector.
\end{enumerate}

\begin{Shaded}
\begin{Highlighting}[]
\NormalTok{A1 }\OtherTok{\textless{}{-}} \FunctionTok{table}\NormalTok{(sp500}\SpecialCharTok{$}\NormalTok{Sector)}
\NormalTok{A2 }\OtherTok{\textless{}{-}} \FunctionTok{as.data.frame}\NormalTok{(A1)}
\FunctionTok{colnames}\NormalTok{(A2) }\OtherTok{\textless{}{-}} \FunctionTok{c}\NormalTok{(}\StringTok{"Sector"}\NormalTok{, }\StringTok{"Stocks"}\NormalTok{)}
\CommentTok{\# Sort the data by number of "Stocks" in descending order}
\NormalTok{A2 }\OtherTok{\textless{}{-}}\NormalTok{ A2[}\FunctionTok{order}\NormalTok{(}\SpecialCharTok{{-}}\NormalTok{A2}\SpecialCharTok{$}\NormalTok{Stocks), ]}
\FunctionTok{kable}\NormalTok{(A2, }\AttributeTok{caption =} \FunctionTok{paste0}\NormalTok{(}\StringTok{"S\&P500 by Sectors, as of "}\NormalTok{, date))}
\end{Highlighting}
\end{Shaded}

\begin{table}

\caption{S&P500 by Sectors, as of 13 September 2023}
\centering
\begin{tabular}[t]{l|l|r}
\hline
  & Sector & Stocks\\
\hline
9 & Finance & 92\\
\hline
17 & Technology Services & 50\\
\hline
7 & Electronic Technology & 49\\
\hline
11 & Health Technology & 47\\
\hline
4 & Consumer Non-Durables & 32\\
\hline
15 & Producer Manufacturing & 31\\
\hline
19 & Utilities & 31\\
\hline
5 & Consumer Services & 29\\
\hline
14 & Process Industries & 24\\
\hline
16 & Retail Trade & 22\\
\hline
8 & Energy Minerals & 16\\
\hline
18 & Transportation & 15\\
\hline
1 & Commercial Services & 13\\
\hline
3 & Consumer Durables & 12\\
\hline
10 & Health Services & 12\\
\hline
6 & Distribution Services & 9\\
\hline
12 & Industrial Services & 9\\
\hline
13 & Non-Energy Minerals & 7\\
\hline
2 & Communications & 3\\
\hline
\end{tabular}
\end{table}

\hypertarget{stock-ratings}{%
\subsection{Stock Ratings}\label{stock-ratings}}

\begin{Shaded}
\begin{Highlighting}[]
\NormalTok{sp500}\SpecialCharTok{$}\StringTok{\textasciigrave{}}\AttributeTok{Technical Rating}\StringTok{\textasciigrave{}} \OtherTok{\textless{}{-}} \FunctionTok{as.factor}\NormalTok{(sp500}\SpecialCharTok{$}\StringTok{\textasciigrave{}}\AttributeTok{Technical Rating}\StringTok{\textasciigrave{}}\NormalTok{)}
\end{Highlighting}
\end{Shaded}

\begin{enumerate}
\def\labelenumi{\arabic{enumi}.}
\setcounter{enumi}{1}
\tightlist
\item
  The levels of the \texttt{Technical\ Rating} are as folows.
\end{enumerate}

\begin{Shaded}
\begin{Highlighting}[]
\FunctionTok{levels}\NormalTok{(sp500}\SpecialCharTok{$}\StringTok{\textasciigrave{}}\AttributeTok{Technical Rating}\StringTok{\textasciigrave{}}\NormalTok{)}
\end{Highlighting}
\end{Shaded}

\begin{verbatim}
[1] "Buy"         "Neutral"     "Sell"        "Strong Buy"  "Strong Sell"
\end{verbatim}

\begin{enumerate}
\def\labelenumi{\arabic{enumi}.}
\setcounter{enumi}{2}
\tightlist
\item
  We count the number of stocks for each level of Rating.
\end{enumerate}

\begin{Shaded}
\begin{Highlighting}[]
\NormalTok{A1 }\OtherTok{\textless{}{-}} \FunctionTok{table}\NormalTok{(sp500}\SpecialCharTok{$}\StringTok{\textasciigrave{}}\AttributeTok{Technical Rating}\StringTok{\textasciigrave{}}\NormalTok{)}
\NormalTok{A2 }\OtherTok{\textless{}{-}} \FunctionTok{as.data.frame}\NormalTok{(A1)}
\FunctionTok{colnames}\NormalTok{(A2) }\OtherTok{\textless{}{-}} \FunctionTok{c}\NormalTok{(}\StringTok{"Rating"}\NormalTok{, }\StringTok{"Stocks"}\NormalTok{)}
\CommentTok{\# Sort the data by number of "Stocks" in descending order}
\CommentTok{\#A2 \textless{}{-} A2[order({-}A2$Stocks), ]}
\FunctionTok{kable}\NormalTok{(A2, }\AttributeTok{caption =} \FunctionTok{paste0}\NormalTok{(}\StringTok{"S\&P500 by Technical Ratings, as of "}\NormalTok{, date))}
\end{Highlighting}
\end{Shaded}

\begin{table}

\caption{S&P500 by Technical Ratings, as of 13 September 2023}
\centering
\begin{tabular}[t]{l|r}
\hline
Rating & Stocks\\
\hline
Buy & 148\\
\hline
Neutral & 62\\
\hline
Sell & 234\\
\hline
Strong Buy & 37\\
\hline
Strong Sell & 22\\
\hline
\end{tabular}
\end{table}

\begin{enumerate}
\def\labelenumi{\arabic{enumi}.}
\setcounter{enumi}{3}
\tightlist
\item
  We count the number of stocks for each level of Rating.
\end{enumerate}

\begin{Shaded}
\begin{Highlighting}[]
\NormalTok{A3 }\OtherTok{\textless{}{-}} \FunctionTok{addmargins}\NormalTok{(}\FunctionTok{table}\NormalTok{(sp500}\SpecialCharTok{$}\NormalTok{Sector, sp500}\SpecialCharTok{$}\StringTok{\textasciigrave{}}\AttributeTok{Technical Rating}\StringTok{\textasciigrave{}}\NormalTok{))}
\FunctionTok{kable}\NormalTok{(A3, }\AttributeTok{caption =} \FunctionTok{paste0}\NormalTok{(}\StringTok{"S\&P500 Sector Stocks by Technical Ratings, as of "}\NormalTok{, date))}
\end{Highlighting}
\end{Shaded}

\begin{table}

\caption{S&P500 Sector Stocks by Technical Ratings, as of 13 September 2023}
\centering
\begin{tabular}[t]{l|r|r|r|r|r|r}
\hline
  & Buy & Neutral & Sell & Strong Buy & Strong Sell & Sum\\
\hline
Commercial Services & 2 & 1 & 9 & 1 & 0 & 13\\
\hline
Communications & 0 & 0 & 3 & 0 & 0 & 3\\
\hline
Consumer Durables & 5 & 0 & 7 & 0 & 0 & 12\\
\hline
Consumer Non-Durables & 7 & 4 & 20 & 0 & 1 & 32\\
\hline
Consumer Services & 8 & 4 & 9 & 7 & 1 & 29\\
\hline
Distribution Services & 0 & 1 & 6 & 0 & 2 & 9\\
\hline
Electronic Technology & 13 & 4 & 29 & 2 & 1 & 49\\
\hline
Energy Minerals & 8 & 3 & 4 & 1 & 0 & 16\\
\hline
Finance & 29 & 13 & 34 & 13 & 3 & 92\\
\hline
Health Services & 3 & 2 & 6 & 0 & 1 & 12\\
\hline
Health Technology & 17 & 8 & 18 & 0 & 4 & 47\\
\hline
Industrial Services & 4 & 2 & 1 & 2 & 0 & 9\\
\hline
Non-Energy Minerals & 2 & 1 & 4 & 0 & 0 & 7\\
\hline
Process Industries & 3 & 4 & 15 & 1 & 1 & 24\\
\hline
Producer Manufacturing & 3 & 3 & 19 & 3 & 3 & 31\\
\hline
Retail Trade & 8 & 3 & 10 & 0 & 1 & 22\\
\hline
Technology Services & 14 & 5 & 27 & 0 & 4 & 50\\
\hline
Transportation & 4 & 2 & 9 & 0 & 0 & 15\\
\hline
Utilities & 18 & 2 & 4 & 7 & 0 & 31\\
\hline
Sum & 148 & 62 & 234 & 37 & 22 & 503\\
\hline
\end{tabular}
\end{table}

This completes our review of Technical Rating.

\hypertarget{create-new-columns}{%
\subsection{Create New Columns}\label{create-new-columns}}

\begin{enumerate}
\def\labelenumi{\arabic{enumi}.}
\tightlist
\item
  \textbf{Low52WkPerc}: Create a new column to track Share Prices
  relative to their 52 Week Low.
\end{enumerate}

\begin{Shaded}
\begin{Highlighting}[]
\NormalTok{sp500 }\OtherTok{\textless{}{-}}\NormalTok{ sp500 }\SpecialCharTok{\%\textgreater{}\%} \FunctionTok{mutate}\NormalTok{(}\AttributeTok{Low52WkPerc =} \FunctionTok{round}\NormalTok{((Price }\SpecialCharTok{{-}} \StringTok{\textasciigrave{}}\AttributeTok{52 Week Low}\StringTok{\textasciigrave{}}\NormalTok{)}\SpecialCharTok{*}\DecValTok{100} \SpecialCharTok{/} \StringTok{\textasciigrave{}}\AttributeTok{52 Week Low}\StringTok{\textasciigrave{}}\NormalTok{,}\DecValTok{2}\NormalTok{))}
\CommentTok{\#colnames(sp500)}
\end{Highlighting}
\end{Shaded}

\begin{enumerate}
\def\labelenumi{\arabic{enumi}.}
\setcounter{enumi}{1}
\tightlist
\item
  \textbf{MarketCapBillions}: Creating a new column MarketCapBillions =
  MarketCap/1000,000,000
\end{enumerate}

\begin{Shaded}
\begin{Highlighting}[]
\NormalTok{sp500 }\OtherTok{\textless{}{-}}\NormalTok{ sp500 }\SpecialCharTok{\%\textgreater{}\%} \FunctionTok{mutate}\NormalTok{(}\AttributeTok{MarketCapBillions =} \StringTok{\textasciigrave{}}\AttributeTok{Market Capitalization}\StringTok{\textasciigrave{}}\SpecialCharTok{/} \DecValTok{1000000000}\NormalTok{)}
\end{Highlighting}
\end{Shaded}

\begin{Shaded}
\begin{Highlighting}[]
\FunctionTok{colnames}\NormalTok{(sp500)}
\end{Highlighting}
\end{Shaded}

\begin{verbatim}
 [1] "Date"                                   
 [2] "Stock"                                  
 [3] "Description"                            
 [4] "Sector"                                 
 [5] "Industry"                               
 [6] "Market Capitalization"                  
 [7] "Price"                                  
 [8] "52 Week Low"                            
 [9] "52 Week High"                           
[10] "Return on Equity (TTM)"                 
[11] "Return on Assets (TTM)"                 
[12] "Return on Invested Capital (TTM)"       
[13] "Gross Margin (TTM)"                     
[14] "Operating Margin (TTM)"                 
[15] "Net Margin (TTM)"                       
[16] "Price to Earnings Ratio (TTM)"          
[17] "Price to Book (FY)"                     
[18] "Enterprise Value/EBITDA (TTM)"          
[19] "EBITDA (TTM)"                           
[20] "EPS Diluted (TTM)"                      
[21] "EBITDA (TTM YoY Growth)"                
[22] "EBITDA (Quarterly YoY Growth)"          
[23] "EPS Diluted (TTM YoY Growth)"           
[24] "EPS Diluted (Quarterly YoY Growth)"     
[25] "Price to Free Cash Flow (TTM)"          
[26] "Free Cash Flow (TTM YoY Growth)"        
[27] "Free Cash Flow (Quarterly YoY Growth)"  
[28] "Debt to Equity Ratio (MRQ)"             
[29] "Current Ratio (MRQ)"                    
[30] "Quick Ratio (MRQ)"                      
[31] "Dividend Yield Forward"                 
[32] "Dividends per share (Annual YoY Growth)"
[33] "Price to Sales (FY)"                    
[34] "Revenue (TTM YoY Growth)"               
[35] "Revenue (Quarterly YoY Growth)"         
[36] "Technical Rating"                       
[37] "Low52WkPerc"                            
[38] "MarketCapBillions"                      
\end{verbatim}

\hypertarget{rename-data-columns}{%
\subsection{Rename Data Columns}\label{rename-data-columns}}

\begin{enumerate}
\def\labelenumi{\arabic{enumi}.}
\item
  The names of the data columns are potentially confusing. We rename the
  data columns to make it easier to work with the data.
\item
  We define a vector of revised column names.
\end{enumerate}

\begin{Shaded}
\begin{Highlighting}[]
\CommentTok{\# Define a mapping of new column names}
\NormalTok{new\_names }\OtherTok{\textless{}{-}} \FunctionTok{c}\NormalTok{(}
  \StringTok{"Date"}\NormalTok{, }\StringTok{"Stock"}\NormalTok{, }\StringTok{"StockName"}\NormalTok{, }\StringTok{"Sector"}\NormalTok{, }\StringTok{"Industry"}\NormalTok{, }
  \StringTok{"MarketCap"}\NormalTok{, }\StringTok{"Price"}\NormalTok{, }\StringTok{"Low52Wk"}\NormalTok{, }\StringTok{"High52Wk"}\NormalTok{, }
  \StringTok{"ROE"}\NormalTok{, }\StringTok{"ROA"}\NormalTok{, }\StringTok{"ROIC"}\NormalTok{, }\StringTok{"GrossMargin"}\NormalTok{, }
  \StringTok{"OperatingMargin"}\NormalTok{, }\StringTok{"NetMargin"}\NormalTok{, }\StringTok{"PE"}\NormalTok{, }
  \StringTok{"PB"}\NormalTok{, }\StringTok{"EVEBITDA"}\NormalTok{, }\StringTok{"EBITDA"}\NormalTok{, }\StringTok{"EPS"}\NormalTok{, }
  \StringTok{"EBITDA\_YOY"}\NormalTok{, }\StringTok{"EBITDA\_QYOY"}\NormalTok{, }\StringTok{"EPS\_YOY"}\NormalTok{, }
  \StringTok{"EPS\_QYOY"}\NormalTok{, }\StringTok{"PFCF"}\NormalTok{, }\StringTok{"FCF"}\NormalTok{, }
  \StringTok{"FCF\_QYOY"}\NormalTok{, }\StringTok{"DebtToEquity"}\NormalTok{, }\StringTok{"CurrentRatio"}\NormalTok{, }
  \StringTok{"QuickRatio"}\NormalTok{, }\StringTok{"DividendYield"}\NormalTok{, }
  \StringTok{"DividendsPerShare\_YOY"}\NormalTok{, }\StringTok{"PS"}\NormalTok{, }
  \StringTok{"Revenue\_YOY"}\NormalTok{, }\StringTok{"Revenue\_QYOY"}\NormalTok{, }\StringTok{"Rating"}\NormalTok{,}
  \StringTok{"Low52WkPerc"}\NormalTok{, }\StringTok{"MarketCapBillions"}
\NormalTok{)}
\end{Highlighting}
\end{Shaded}

\begin{enumerate}
\def\labelenumi{\arabic{enumi}.}
\setcounter{enumi}{2}
\tightlist
\item
  We rename the columns using the \texttt{colnames()} or
  \texttt{names()} function.
\end{enumerate}

\begin{Shaded}
\begin{Highlighting}[]
\CommentTok{\# Rename the columns }
\FunctionTok{colnames}\NormalTok{(sp500) }\OtherTok{\textless{}{-}}\NormalTok{ new\_names}
\CommentTok{\#sp500 \textless{}{-} sp500 \%\textgreater{}\% rename\_all(\textasciitilde{}new\_names)}
\end{Highlighting}
\end{Shaded}

\begin{Shaded}
\begin{Highlighting}[]
\FunctionTok{colnames}\NormalTok{(sp500)}
\end{Highlighting}
\end{Shaded}

\begin{verbatim}
 [1] "Date"                                   
 [2] "Stock"                                  
 [3] "Description"                            
 [4] "Sector"                                 
 [5] "Industry"                               
 [6] "Market Capitalization"                  
 [7] "Price"                                  
 [8] "52 Week Low"                            
 [9] "52 Week High"                           
[10] "Return on Equity (TTM)"                 
[11] "Return on Assets (TTM)"                 
[12] "Return on Invested Capital (TTM)"       
[13] "Gross Margin (TTM)"                     
[14] "Operating Margin (TTM)"                 
[15] "Net Margin (TTM)"                       
[16] "Price to Earnings Ratio (TTM)"          
[17] "Price to Book (FY)"                     
[18] "Enterprise Value/EBITDA (TTM)"          
[19] "EBITDA (TTM)"                           
[20] "EPS Diluted (TTM)"                      
[21] "EBITDA (TTM YoY Growth)"                
[22] "EBITDA (Quarterly YoY Growth)"          
[23] "EPS Diluted (TTM YoY Growth)"           
[24] "EPS Diluted (Quarterly YoY Growth)"     
[25] "Price to Free Cash Flow (TTM)"          
[26] "Free Cash Flow (TTM YoY Growth)"        
[27] "Free Cash Flow (Quarterly YoY Growth)"  
[28] "Debt to Equity Ratio (MRQ)"             
[29] "Current Ratio (MRQ)"                    
[30] "Quick Ratio (MRQ)"                      
[31] "Dividend Yield Forward"                 
[32] "Dividends per share (Annual YoY Growth)"
[33] "Price to Sales (FY)"                    
[34] "Revenue (TTM YoY Growth)"               
[35] "Revenue (Quarterly YoY Growth)"         
[36] "Technical Rating"                       
[37] "Low52WkPerc"                            
[38] "MarketCapBillions"                      
\end{verbatim}



\end{document}
